\section{Conclusions}

The actual and final result which was achieved is not perfect, but a great step towards a running "Micromouse" was accomplished. From the fact that no team member had proficiency in electrical engineering, when this project was started, one can say, that by dividing this challenge into three groups the result couldn't have been better (regarding the time constraint). \\
All in all, the project was a success regarding our lack of knowledge and by comparing it to the outcome. Also, a lot of new skills were learned by all team members.\\

\noindent
As the challenge was divided into three groups - Hardware, Software and Casing - one clearly can see the resulted outcome:
\begin{itemize}
    \item a printed PCB
    \item all ordered electrical components
    \item a working software
    \item a 3D-printed case which holds the PCB, both motors and battery\\
\end{itemize}

\noindent
These are the next steps which have to be done:
\begin{itemize}
    \item soldering of all electrial components onto the PCB
    \item feeding the software into the microcontroller
    \item testing and adjusting of the sensors for front, left and right
    \item writing the algorithm to find the center of a maze
\end{itemize}

\subsection{Expectations}
\textcolor{red}{
Slightly controversial part, we could omit it (I would like not to though) or rephrase it somehow. Basically - what did we expect from the course. Not sure if this part actually belongs here, but for now it works.
}
\subsection{Proposals}

\begin{itemize}
    \item It would have been great to have a reference recommendation for everything that came up along the way. We know that we got our hands on many different subjects and issues. Even though, it would be nice to provide a reference for everything, from book chapters to articles or papers. For example, the document provided in the beginning was an excellent start, which unfortunately didn't cover much of electronics. This would help inexperienced students catch up with some extra effort and prove to be invaluable for referencing in the final report. All in all, I would use the book "Art of Electronics" for specific referencing.
    \item Have a presentation in the middle of the term. We believe the presentation pushed everyone and we really understood what is going on and what the progress is. We believe a midterm evaluation of the progress would help motivate more work and identify still missing points of the project.
    \item For students who don't have the knowledge yet to build a complete robot from scratch, it would have been nice to have more options to choose from. Although the course's name specifies that a robot is going to be "build from scratch", one still could introduce two kind of groups: One group which is building the hardware from scratch and another group which is responsible for the software and trying to optimize the search algorithm (by using a ready-to-use Micromouse).
\end{itemize}