\section{Conclusions}
\textcolor{red}{
Here - short summary of the achieved results, maybe some general words and praises for our final mouse version, just something positive to end the mouse story well.
}\\

\noindent
The actual and final result we have achieved is not perfect, but we accomplished a great step towards a running "Micromouse". From the fact that no team member had profiency in electrical engineering, as we started this project, one can say, that we've learned a lot of new skills. \\
All in all, the project was a success regarding our lack of knowledge and comparing it to the outcome. 

\subsection{Expectations}
\textcolor{red}{
Slightly controversial part, we could omit it (I would like not to though) or rephrase it somehow. Basically - what did we expect from the course. Not sure if this part actually belongs here, but for now it works.
}
\subsection{Proposals}

\textcolor{red}{We could rephrase the following as personal suggestions or as team, if everyone agrees.}\\
\textcolor{blue}{I would phrase them as a team. We might have different opinions (but this belongs into a feedback, we can write him via mail) but in the report we should unify them.}

Adam: It would have been great to have a reference recommendations for everything that came up along the way. I know that we got our hands on many different subjects and issues. Even though Alex is a great and experienced mentor on electronics, it would be nice to provide a reference for everything, from book chapters to articles or papers. For example, the document provided in the beginning was an excellent start, which unfortunately didn't cover much of electronics. This would help inexperienced students catch up with some extra effort and prove to be invaluable for referencing in the final report. All in all, I would use the book Art of Electronics for specific referencing.

Adam: Another proposal. Have a presentation in the middle of the term. I believe the presentation pushed everyone and we really understood what is going on and what the progress is. I personally believe a midterm evaluation of the progress would help motivate more work and identify still points of the project.

René: For students who don't have the knowledge yet to build a complete robot from scratch (like us), it would have been nice to have more options. Although the course's name specifies that a robot is going to be build from scratch, one still could introduce two kind of groups: One group which is building the hardware from scratch and another group which is responsible for the software and trying to optimize the search algorithm (by using a ready-to-use Micromouse).