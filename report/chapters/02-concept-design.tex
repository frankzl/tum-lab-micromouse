\section{Conceptual design and justification of the design}

\subsection{Initial design conditions}
    Here we talk about the size of the labyrinth, explain the size of the mouse (maybe touch on our initial concept of the mouse design or something similar? - later we could extend this topic and motivate our casing design decisions, obviously mentioning the Fusion workshop, but this comes later)
    Also here should come some innate design "choices" like using a microcontroller, a number of sensors, why we need those e.t.c
    Basically the whole logic behind "how do we build a robot that should drive in the labyrinth and be able to make intelligent decisions (turns) based on the observations (made by sensors)
    Maybe we could split this section into two subsections:\\
    - given conditions\\
    - and our design decisions based on those conditions (justification)
    
\subsection{Work program and Gantt chart}
This is where I'm a bit lost. We could include some "ideal" version of this chart here, but where exactly should we describe all the changes in planning and organization and why they had to be done at each stage of building the prototype and the final version of the mouse?  Should it be described here? Or later in the "problems and challenges"?
Also I think maybe here we should talk about all the initial learning stage we had to go through in the first half of the praktikum (maybe devote a subsection for this here or later in the report)

